\documentclass[12pt]{article}
% Pachete necesare
\usepackage[a4paper, margin=1in]{geometry}
\usepackage{amsmath}
\usepackage{cite}
\usepackage{hyperref}
\usepackage{setspace}
\onehalfspacing
% \usepackage{microtype} % Eliminat pentru a evita avertismentul
\usepackage{newtxtext}
\usepackage[cmintegrals]{newtxmath}
\usepackage{graphicx}
\usepackage{tikz}
\usetikzlibrary{arrows.meta, decorations.pathmorphing}
\usepackage{pgfplots}
\pgfplotsset{compat=1.18}
\hyphenation{de-lib-er-ate fluc-tu-a-tions tem-per-a-ture ob-serv-er cog-ni-tive in-flu-ence sim-u-la-tion fa-tigue sta-tis-ti-cal de-vi-a-tion}
\sloppy
\usepackage{titling}
\setlength{\droptitle}{-2cm}
\setlength{\textfloatsep}{0.5cm}
\setlength{\intextsep}{0.5cm}

% Titlu, autor
\title{Cognitive Signals Influencing Quantum Systems: A Biological Hypothesis}
\author{Teodor Berger \\ Independent Researcher}
\date{18 May 2025}

\begin{document}

\maketitle

\begin{abstract}
This study hypothesizes that human cognition emits subtle signals that perturb quantum systems, potentially explaining the observer effect. A double-slit experiment tests this by comparing photon interference patterns with and without a human subject. The research bridges quantum mechanics and cognitive science, offering new perspectives on consciousness.
\end{abstract}

\section{Introduction}
The observer effect in quantum mechanics, where measurement alters a system's state, has puzzled researchers for decades \cite{wheeler1978law}. The double-slit experiment illustrates this: particles show wave-like interference when unobserved but particle-like behavior when measured. The role of human consciousness in this process remains debated \cite{zeilinger1999foundations}. This paper proposes that involuntary cognitive signals influence quantum outcomes, tested through a novel experiment. The study aims to connect quantum physics with cognitive science, with implications for both fields.

\section{Theoretical Background}
We propose the "cognitive radar hypothesis," inspired by biological signaling like bat echolocation \cite{penrose1989emperors}. Human cognition may emit quantum-level signals, distinct from electromagnetic waves, that perturb particle behavior. Neural activity, with its synchronized patterns, could generate a field-like effect, active during wakefulness but reduced in sleep. This hypothesis reframes the observer effect as a biological phenomenon, drawing on interdisciplinary insights from quantum mechanics and neuroscience.

\section{Experimental Methodology}
A double-slit experiment tests the hypothesis, detailed below.

\subsection{Experimental Setup}
A photon source emits paired particles via spontaneous parametric down-conversion. Two slits, 0.1 mm apart, are 1 m from a detector screen. A human subject, unaware of emission timing, is seated 2 m away. The setup is shielded to minimize noise.

\subsection{Procedures and Controls}
We compare:
\begin{itemize}
    \item \textbf{Presence}: Subject is awake and present.
    \item \textbf{Absence}: No subject or subject in meditation.
\end{itemize}
Each condition runs 1000 trials. Errors (cognitive variability, environmental noise, fatigue) are mitigated by multiple subjects, shielding, and short sessions.

\subsection{Data Analysis}
\begin{sloppypar}
Photon positions form a histogram. Fringe variance is computed, and a t-test compares conditions (p=0.05).
\end{sloppypar}

\begin{figure}[!ht]
\vspace{0.5cm}
\centering
\resizebox{0.7\textwidth}{!}{%
\begin{tikzpicture}[scale=0.9]
    % Sursa de fotoni
    \draw[fill=gray!50] (-3,0) circle (0.3) node[left, font=\normalsize, xshift=-0.5cm] {Photon Source};
    % Fante
    \draw[line width=1.2pt] (0,-1) -- (0,1) node[above, font=\normalsize] {Slit 1};
    \draw[line width=1.2pt] (1,-1) -- (1,1) node[above, font=\normalsize] {Slit 2};
    % Ecran detector
    \draw[line width=1.2pt] (5,-2) -- (5,2) node[midway, right, font=\normalsize, xshift=0.5cm] {Detector Screen};
    \node[font=\normalsize, rotate=90] at (5.5,0) {Interference Pattern};
    % Subiect uman
    \draw[fill=gray!50] (-2,-3.5) circle (0.4) node[below, font=\normalsize] {Human Subject};
    % Traiectorii fotonice
    \draw[->>, dashed, line width=1pt] (-3,0) -- (0,0.5);
    \draw[->>, dashed, line width=1pt] (-3,0) -- (1,-0.5);
    \draw[->>, dashed, line width=1pt] (0,0.5) -- (5,1);
    \draw[->>, dashed, line width=1pt] (1,-0.5) -- (5,-1);
    % Semnal cognitiv
    \draw[->, red!60!black, decorate, decoration={snake, amplitude=0.5mm}, line width=1pt] (-2,-3.5) -- (0.5,0);
    \node[font=\normalsize, red!60!black] at (0.5,0.7) {Cognitive Signal};
\end{tikzpicture}%
}
\caption{Experimental setup: a human subject's cognitive signals may influence photon interference patterns in a double-slit experiment.}
\label{fig:setup}
\vspace{0.5cm}
\end{figure}

\section{Numerical Simulations}
A numerical model simulates interference patterns, with a perturbation term for cognitive influence. Code is available at \url{https://github.com/DonMask/QuantumCognitiveRadar} (\texttt{simulation.py}), archived at \href{https://doi.org/10.5281/zenodo.15458571}{Zenodo (DOI: 10.5281/zenodo.15458571)}. Figure~\ref{fig:interference} shows a shifted pattern with a subject present.

\begin{figure}[!ht]
\vspace{0.5cm}
\centering
\resizebox{0.7\textwidth}{!}{%
\begin{tikzpicture}
\begin{axis}[
    xlabel={Position on Detector Screen (mm)},
    ylabel={Photon Count},
    xlabel style={font=\large},
    ylabel style={font=\large},
    domain=-5:5,
    samples=100,
    width=\textwidth,
    height=5cm,
    grid=major,
    grid style={opacity=0.5},
    legend pos=north east,
    legend style={font=\large}
]
    \addplot[blue!70!black, thick, line width=1.5pt] {100*(cos(deg(x)))^2};
    \addlegendentry{No Subject}
    \addplot[red!80!black, dashed, line width=1.5pt] {80*(cos(deg(x)))^2 + 20};
    \addlegendentry{Subject Present}
\end{axis}
\end{tikzpicture}%
}
\caption{Simulated interference patterns: blue curve (no human subject), red dashed curve (human subject present).}
\label{fig:interference}
\vspace{0.5cm}
\end{figure}

\section{Mathematical Model}
The interference pattern is modeled as:
\begin{equation}
P(x) = \left| \psi_1(x) + \psi_2(x) \right|^2,
\end{equation}
where \(\psi_1(x)\), \(\psi_2(x)\) are slit wavefunctions. A cognitive phase shift \(\phi\) yields:
\begin{equation}
P'(x) = \left| \psi_1(x)e^{i\phi} + \psi_2(x) \right|^2.
\end{equation}
This predicts measurable fringe shifts.

\section{Discussions and Implications}
If validated, the hypothesis could inform quantum sensor design and neuroscience \cite{penrose1989emperors}. Challenges include detecting subtle signals and isolating cognitive effects. Future studies could explore signal characteristics or different mental states, advancing quantum-cognitive research.

\section{Conclusion}
This study proposes that cognitive signals perturb quantum systems, tested via a double-slit experiment. Validation could bridge quantum mechanics and cognitive science, prompting new research directions.

\clearpage
\begin{thebibliography}{1}
\bibitem{wheeler1978law}
J.~A. Wheeler, ``The `law without law','' in \emph{Quantum Theory and Measurement}, J.~A. Wheeler and W.~H. Zurek, Eds., Princeton University Press, 1978.
\bibitem{zeilinger1999foundations}
A.~Zeilinger, ``Foundations of quantum mechanics,'' \emph{Physics World}, vol.~12, no.~4, pp.~35--40, 1999.
\bibitem{penrose1989emperors}
R.~Penrose, \emph{The Emperor's New Mind}, Oxford University Press, 1989.
\end{thebibliography}

\end{document}
